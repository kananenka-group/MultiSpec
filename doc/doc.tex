\documentclass{article}

\usepackage{amsmath,amssymb}
\usepackage[superscript]{cite}
%\usepackage{cleveref}
\usepackage{tikz}
\usetikzlibrary{arrows.meta}
\usepackage{hyperref}
\hypersetup{
    colorlinks=true,
    linkcolor=blue,
    filecolor=magenta,      
    urlcolor=black,
    pdfpagemode=FullScreen,
    }

\renewcommand\labelitemi{-{}-}

\begin{document}

\title{MultiSpec reference guide}
\author{Alexei A. Kananenka}
\date{\today}
\maketitle
\tableofcontents

\section{Installation}
MultiSpec can be downloaded and installed using cmake:
\begin{verbatim}
$ git clone https://github.com/kananenka-group/MultiSpec.git
$ cd ./MultiSpec/src
$ mkdir build 
$ cmake -S . -B build
$ cmake --build build
\end{verbatim}

Note that MultiSpec uses Boost and Blas, both are needed to be installed before attempting to install MultiSpec.

\section{Exciton module}
Exciton module is the main module that computes spectra. Currently only linear IR and Raman spectra are implemented.

\subsection{Input parameters}

\begin{itemize}
\item \textbf{dt}: (\texttt{double}) time step between frames in ps.
\item \textbf{tc}: (\texttt{double}) correlation time for 1D time-correlation functions.
\item \textbf{H}: (\texttt{string}) name of the file containing Hamiltonian trajectory. 
\item \textbf{D}: (\texttt{string}) name of the file containing transition dipole trajectory. 
\item \textbf{P}: (\texttt{string}) name of the file containing transition polarizability trajectory. 
\item \textbf{IR}: (\texttt{bool}) calculate linear IR spectra, options: \{1,0\}.
\item \textbf{Raman}: (\texttt{bool}) calculate Raman spectra, options: \{1,0\}. This will calculate VV, VH, isotropic, and unpolarized Raman spectra.
\item \textbf{SFG}: (\texttt{bool}) calculate sum-frequency generation (SFG) spectra, options: \{1,0\}. This will calculate SFG in \textit{ssp} polarization.
\item \textbf{nframes}: (\texttt{int}) how many frames are stored in trajectory files.
\item \textbf{T1}: (\texttt{double}) T$_1$ time, life-time of the first exited state.
\item \textbf{navg}: (\texttt{int}) the number of segments the input trajectories will be divided into and used for statistical averaging.
\item \textbf{tstart}: (\texttt{double}) Starting time for calculating spectra (ps). Rewind the trajectory to this time and start calculating spectra. 
\item \textbf{tsep}: (\texttt{double}) time separation in ps between segments.
\item \textbf{w\_avg}: (\texttt{double}) (optional parameter) average frequency  in cm$^{-1}$. It helps mitigate numerical instabilities.
\item \textbf{inh}: (\texttt{bool}) calculate spectra in the inhomogeneous broadening limit. This also requests
calculation of distribution of excitonic frequencies.

\end{itemize}

\section{Water module}
Water module generates an input for the exciton module. The input consists of excitonic hamiltonian trajectory as well as 
transition dipole and transition polarizability trajectories for IR, Raman, and SFG calculations.

\subsection{Input parameters}

\begin{itemize}

\item \textbf{xtc}: (\texttt{string}) path to gromacs *.xtc file.
\item \textbf{gro\_file}: (\texttt{string}) path to gromacs *.gro file.
\item \textbf{start}: (\texttt{int}) starting frame to read from the trajectory (xtc file).
\item \textbf{IR}: (\texttt{bool}) calculate transition dipole derivative moments for IR spectra, options: \{1,0\}.
\item \textbf{Raman}: (\texttt{bool}) calculate transition polarizability trajectories for Raman spectra, options: \{1,0\}.
\item \textbf{SFG}: (\texttt{bool}) calculate transition polarizability and transition dipole trajectories for SFG spectra, options: \{1,0\}.
\item \textbf{nframes}: (\texttt{int}) how many frames read from *.xtc file and process.
\item \textbf{atoms\_file}: (\texttt{string}) path to the file containing charges and masses of all atoms. This is a simple three-column file containing atom names matching atoms in 
*.gro file, corresponding charges and masses. Comment lines must start with \textsc{\#}. An example of such file can be found in Sec.~\ref{app:atomsf}.
\item \textbf{water\_model}: (\texttt{string}) water model. Supported water models: \texttt{SPC}, \texttt{SPCE}, \texttt{TIP4P}, \texttt{TIP4P/2005}, \texttt{E3B2}, \texttt{E3B3}.
Note that in each case \texttt{atoms\_file} is required and must have charges and masses of all atoms of the model making it possible to use any of these models
with any desired charges.
\item \textbf{stretch\_map}: (\texttt{string}) spectroscopic map for OH and OD stretch. The following maps have been implemented so far:
 \texttt{li\_2010\_tip4p},  \texttt{gruenbaum\_2013\_tip4p}, and \texttt{auer\_2008\_spce}. See Sec.~\ref{sec:wmaps} for more details.

\item \textbf{bend\_map}: (\texttt{string}) spectroscopic map for HOH and DOD bend. 
The following maps have been implemented so far:
\begin{itemize}
\item \texttt{ni\_2015\_tip4p} from Y. Ni and J. L. Skinner, J. Chem. Phys. 143, 014502 (2015)
\item  \texttt{ni\_2015\_kananenka\_2019\_tip4p} from Y. Ni and J. L. Skinner , J. Chem. Phys. 143, 014502 (2015) updated with HOD and D$_2$O 
bending frequencies used in Kananenka \textit{et al.}, J. Phys. Chem. B 123, 5139-5146 (2019).
\end{itemize}


\item \textbf{spec\_type}: (\texttt{string}) type of calculation that will be performed. Supported types:
\begin{itemize}
\item OH stretch, keyword: \texttt{wsOH}
\item OD stretch, keyword: \texttt{wsOD}
\item hydroxyl stretch in water isotope mixtures, keyword: \texttt{wsiso}
\item OH-stretch fundamental-HOH bend overtone, keyword: \texttt{wswbH2O}
\item OD-stretch fundamental-DOD bend overtone, keyword: \texttt{wswbD2O}
\item hydroxyl stretch fundamental-bend overtone in water isotope mixtures, keyword: \texttt{wswbiso}.
\item uncoupled OH-stretch fundamental, keyword: \texttt{wuOH}
\item uncoupled OD-stretch fundamental, keyword: \texttt{wuOD}
\end{itemize}

\item \textbf{D2O}: (\texttt{int}) the number of D$_2$O molecules mixed with H$_2$O. This is only required for \texttt{--spec\_type=wsiso}
and  \texttt{--spec\_type=wswbiso}.

\item \textbf{Fc}: (\texttt{float}) OH-stretch fundamental-HOH bend overtone Fermi coupling. This is  required only for \texttt{--spec\_type=wswbH2O},
\texttt{--spec\_type=wswbD2O}, and \texttt{--spec\_type=wswbiso}.

\item \textbf{DOD\_overtone}: (\texttt{bool}) use this option to turn on/off DOD bend overtone in water hydroxyl stretch fundamental/bend overtone
calculations of isotope mixtures. This option only works with \texttt{--spec\_type=wswbiso}.

\item \textbf{trdipSFG}: (\texttt{float}) distance in \AA{} between O atom and the
OH-stretch transition dipole along the O-H(D) bond. This only works with  
\texttt{--SFG=1}.

\item \textbf{intrac}: (\texttt{bool}) turns on and off OH stretch intramolecular couplings, options: \{1,0\}. Default value is 1.

\item \textbf{intercOH}: (\texttt{bool}) turns on and off OH stretch intermolecular couplings, options: \{1,0\}. Default value is 1.

\item \textbf{exc\_ham}: (\texttt{bool}) if set to 1 will print diagonal frequencies, intermolecular, and intramolecular couplings to
separate files. Options: \{1,0\}. Default value is 0.

\end{itemize}

\subsection{\label{sec:wmaps} Water Spectroscopic maps}
\subsubsection{gruenbaum\_2013\_tip4p}
These maps were developed by Gruenbaum et al. in \href{https://pubs.acs.org/doi/10.1021/ct400292q}{J. Chem. Theory Comput. 9, 3109 (2013)} based on clusters
of water molecules from a liquid-state simulation of TIP4P water. Note that some of the maps shown below have been
developed before this paper.
OH stretching maps:
\begin{eqnarray}
\omega_{10} &=& 3760.2 - 3541.7E-152677E^2 \nonumber\\
\omega_{21} &=& 3606.0 - 3498.6E - 198715E^2 \nonumber \\
x_{10} &=& 0.19285 - 1.7261\cdot 10^{-5} \omega_{10} \nonumber \\
x_{21} &=& 0.26836 - 2.3788\cdot 10^{-5} \omega_{21} \nonumber \\
p_{10} &=& 1.6466 + 5.7692\cdot 10^{-4} \omega_{10} \nonumber \\
p_{21} &=& 2.0160 + 8.7684\cdot 10^{-4} \omega_{21} \nonumber 
\end{eqnarray}
OD stretching maps:
\begin{eqnarray}
\omega_{10} &=& 2767.8 - 2630.3E-102601E^2 \nonumber\\
\omega_{21} &=& 2673.0 - 1763.5E - 138534E^2 \nonumber \\
x_{10} &=& 0.16593 - 2.0632\cdot 10^{-5} \omega_{10} \nonumber \\
x_{21} &=& 0.23167 - 2.8596\cdot 10^{-5} \omega_{21} \nonumber \\
p_{10} &=& 2.0475 + 8.9108\cdot 10^{-4} \omega_{10} \nonumber \\
p_{21} &=& 2.6233 + 13.1443\cdot 10^{-4} \omega_{21} \nonumber 
\end{eqnarray}
Dipole derivative and intramolecular coupling maps:
\begin{eqnarray}
\mu' &=& 0.1646 + 11.39E + 63.41E^2 \nonumber \\
\omega_{jk}^\text{intra}&=& [-1361 + 27165(E_j + E_k)]x_jx_k - 1.887 p_j p_k \nonumber
\end{eqnarray}
OH/OD stretch transition dipole is located 0.67 \AA{} away from the water oxygen atom along the OH/OD bond.


\subsubsection{li\_2010\_tip4p}
These maps were developed by Li and Skinner in \href{https://pubs.acs.org/doi/10.1021/ct400292q}{J. Chem. Phys. 132, 244504 (2010)} based on clusters
of water molecules from a liquid-state simulation of TIP4P water. OH stretching maps:
\begin{eqnarray}
\omega_{10} &=& 3732.9 - 3519.8E-1.5353\cdot 10^5E^2 \nonumber\\
x_{10} &=& 0.19318 - 1.7248\cdot 10^{-5} \omega_{10} \nonumber \\
p_{10} &=& 1.6120 + 5.8697\cdot 10^{-4} \omega_{10} \nonumber \\
\end{eqnarray}
OD stretching maps:
\begin{eqnarray}
\omega_{10} &=& 2748.2 - 2572.2E-1.0298\cdot 10^5 E^2 \nonumber\\
x_{10} &=& 0.16598 - 2.0752\cdot 10^{-5} \omega_{10} \nonumber \\
p_{10} &=& 1.9813 + 9.1419\cdot 10^{-4} \omega_{10} \nonumber \\
\end{eqnarray}
Dipole derivative and intramolecular coupling maps:
\begin{eqnarray}
\mu' &=& 0.1622 + 10.381E + 137.6E^2 \nonumber \\
\omega_{jk}^\text{intra}&=& [-1361 + 27165(E_j + E_k)]x_jx_k - 1.887 p_j p_k \nonumber
\end{eqnarray}
OH/OD stretch transition dipole is located 0.67 \AA{} away from the water oxygen atom along the OH/OD bond.

\subsubsection{auer\_2008\_spce}
These maps were developed by  Auer et al. in \href{https://doi.org/10.1063/1.2925258}{J. Chem. Phys. 128, 224511 (2008)} 
and \href{http://dx.doi.org/10.1063/1.3409561}{J. Chem. Phys. 132, 174505 (2010)} (OD $\omega_{21}$ and $x_{21}$ maps)
and \href{http://www.pnas.org/cgi/doi/10.1073/pnas.0701482104}{PNAS 104, 14215-4220} (OH $\omega_{21}$ and $x_{21}$ maps)
based on clusters
of water molecules from a liquid-state simulation of SPC/E water. OH stretching maps:
\begin{eqnarray}
\omega_{10} &=& 3761.6 - 5060.4 E - 86225 E^2\nonumber\\
x_{10} &=& 0.1934 - 1.75\cdot 10^{-5} \omega_{10} \nonumber \\
p_{10} &=& 1.611 + 5.893\cdot 10^{-4} \omega_{10} \nonumber \\
\omega_{21} &=& 3614.1 - 5493.7E-115670 E^2 \nonumber \\
x_{21} &=& 0.1428 - 1.29\cdot 10^{-5} \omega_{21} \nonumber 
\end{eqnarray}
OD stretching maps:
\begin{eqnarray}
\omega_{10} &=& 2762.6 - 3640.8E-56641 E^2 \nonumber\\
x_{10} &=& 0.16627 - 2.0884\cdot 10^{-5} \omega_{10} \nonumber \\
p_{10} &=& 1.9844 + 9.1907\cdot 10^{-4} \omega_{10} \nonumber \\
\omega_{21} &=& 2695.8 - 3785.1E-73074 E^2 \nonumber \\
x_{21} &=& 0.1229 - 1.525\cdot 10^{-5} \omega_{21} \nonumber 
\end{eqnarray}
Dipole derivative and intramolecular couplings:
\begin{eqnarray}
\mu' &=& 0.1333 + 14.17E \nonumber \\
\omega_{jk}^\text{intra}&=& [-1789 + 23852(E_j + E_k)]x_jx_k - 1.966 p_j p_k \nonumber
\end{eqnarray}
OH/OD stretch transition dipole is located 0.58 \AA{} away from the water oxygen atom along the OH/OD bond.

%%%%
%
%. AMIDE I
%
%%%%%
\section{AmideI module}
\subsection{Input parameters}

\begin{itemize}

\item \textbf{xtc}: (\texttt{string}) path to gromacs *.xtc file.
\item \textbf{gro\_file}: (\texttt{string}) path to gromacs *.gro file.
\item \textbf{top\_file}: (\texttt{string}) path to gromacs *.top file.
%\item \textbf{start}: (\texttt{int}) starting frame to read from the trajectory (xtc file).
%\item \textbf{IR}: (\texttt{bool}) calculate transition dipole derivative moments for IR spectra, options: \{1,0\}.
%\item \textbf{Raman}: (\texttt{bool}) calculate transition polarizability trajectories for Raman spectra, options: \{1,0\}.
%\item \textbf{SFG}: (\texttt{bool}) calculate transition polarizability and transition dipole trajectories for SFG spectra, options: \{1,0\}.
\item \textbf{itp\_files}: (\texttt{vector<string>}) path to GROMACS topology files (*.itp) containing residue information and charges and masses of all atoms. Provide a list of files if more than one file contains the topology info about the whole system.
\item \textbf{nframes}: (\texttt{int}) how many frames read from *.xtc file and process.
\item \textbf{spec\_type}: (\texttt{string}) type of calculation that will be performed. Supported types:
\begin{itemize}
\item Amide I spectroscopy with all amide I groups included (no isotope labels), keyword: \texttt{full}
\item Isotope labels placed at selected residues, keyword: \texttt{iso}
\end{itemize}
\item \textbf{isotope\_labels}: (\texttt{vector<string>}) provide a list of residues whose C=O groups are isotope labeled. Example: ``22SER 23ALA...".
\item \textbf{nn\_map}: (\texttt{string}) Nearest-neighbor map diagonal frequency map. Currently implemented: \texttt{Jansen\_2006} map.
\item \textbf{isotope\_shift}: (\texttt{float}) isotope frequency shift. The default value is $-$66 cm$^{-1}$ corresponding to $^{13}$C$^{18}$O isotope shift
taken from JACS 134, 19118-19128 (2012).
\end{itemize}


\section{Examples}


 



\begin{appendix}
\section{\label{app:atomsf}atoms\_file}
Below is an example of a file that can be used with the \textbf{atoms\_file} command line. This is a pure water TIP4P simulation:
\begin{verbatim}
# this is a comment line; atom  charge  mass
OW      0.0     16.000
HW1     0.52    1.008
HW2     0.52    1.008
MW      -1.04   0.000
\end{verbatim}

\end{appendix}

\end{document}