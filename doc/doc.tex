\documentclass{article}

\usepackage{amsmath,amssymb}
\usepackage[superscript]{cite}
\usepackage{cleveref}
\usepackage{tikz}
\usetikzlibrary{arrows.meta}

\renewcommand\labelitemi{-{}-}

\begin{document}

\title{MultiSpec reference guide}
\author{Alexei A. Kananenka}
\date{\today}
\maketitle
\tableofcontents

\section{Exciton module}
Exciton module is the main module that computes spectra. Currently only linear IR and Raman spectra are implemented.

\subsection{Input parameters}

\begin{itemize}
\item \textbf{dt}: (\texttt{double}) time step between frames in ps.
\item \textbf{tc}: (\texttt{double}) correlation time for 1D time-correlation functions.
\item \textbf{H}: (\texttt{string}) name of the file containing Hamiltonian trajectory. 
\item \textbf{D}: (\texttt{string}) name of the file containing transition dipole trajectory. 
\item \textbf{P}: (\texttt{string}) name of the file containing transition polarizability trajectory. 
\item \textbf{IR}: (\texttt{bool}) calculate linear IR spectra, options: \{1,0\}.
\item \textbf{Raman}: (\texttt{bool}) calculate Raman spectra, options: \{1,0\}. This will calculate VV, VH, isotropic, and unpolarized Raman spectra.
\item \textbf{SFG}: (\texttt{bool}) calculate sum-frequency generation (SFG) spectra, options: \{1,0\}. This will calculate SFG in \textit{ssp} polarization.
\item \textbf{nframes}: (\texttt{int}) how many frames are stored in trajectory files.
\item \textbf{T1}: (\texttt{double}) T$_1$ time, life-time of the first exited state.
\item \textbf{navg}: (\texttt{int}) the number of segments the input trajectories will be divided into and used for statistical averaging.
\item \textbf{tsep}: (\texttt{double}) time separation in ps between segments.
\item \textbf{w\_avg}: (\texttt{double}) average frequency (optional) helps mitigate numerical instabilities.

\end{itemize}

\section{Water module}
Water module generates an input for the exciton module. The input consists of excitonic hamiltonian trajectory as well as 
transition dipole and transition polarizability trajectories for IR, Raman, and SFG calculations.

\subsection{Input parameters}

\begin{itemize}

\item \textbf{xtc}: (\texttt{string}) path to gromacs *.xtc file.
\item \textbf{gro\_file}: (\texttt{string}) path to gromacs *.gro file.
\item \textbf{IR}: (\texttt{bool}) calculate transition dipole derivative moments for IR spectra, options: \{1,0\}.
\item \textbf{Raman}: (\texttt{bool}) calculate transition polarizability trajectories for Raman spectra, options: \{1,0\}.
\item \textbf{SFG}: (\texttt{bool}) calculate transition polarizability and transition dipole trajectories for SFG spectra, options: \{1,0\}.
\item \textbf{nframes}: (\texttt{int}) how many frames read from *.xtc file and process.
\item \textbf{atoms\_file}: (\texttt{string}) path to the file containing charges and masses of all atoms. This is a simple three-column file containing atom names matching atoms in 
*.gro file, corresponding charges and masses. Comment lines must start with \textsc{\#}. An example of such file can be found in Sec.~\ref{app:atomsf}.
\item \textbf{water\_model}: (\texttt{string}) water model. Supported water models: \texttt{SPC}, \texttt{SPC/E}, \texttt{TIP4P}, \texttt{TIP4P/2005}, \texttt{E3B2}, \texttt{E3B3}.
Note that in each case \texttt{atoms\_file} is required and must have charges and masses of all atoms of the model making it possible to use any of these models
with any desired charges.
\item \textbf{stretch\_map}: (\texttt{string}) spectroscopic map for OH and OD stretch. The following maps have been implemented so far:
\begin{itemize}
\item \texttt{li\_2010\_tip4p} from F. Li and J. L. Skinner, J. Chem. Phys. 132, 244504 (2010) 
\item  \texttt{gruenbaum\_tip4p\_2013} from S. M. Gruenbaum et al., J. Chem. Theory Comput. 9, 3109 (2013)
\end{itemize}

\item \textbf{bend\_map}: (\texttt{string}) spectroscopic map for HOH and DOD bend. 
The following maps have been implemented so far:
\begin{itemize}
\item \texttt{ni\_2015\_tip4p} from Y. Ni and J. L. Skinner, J. Chem. Phys. 143, 014502 (2015)
\item  \texttt{ni\_2015\_kananenka\_2019\_tip4p} from Y. Ni and J. L. Skinner , J. Chem. Phys. 143, 014502 (2015) updated with HOD and D$_2$O 
bending frequencies used in Kananenka \textit{et al.}, J. Phys. Chem. B 123, 5139-5146 (2019).
\end{itemize}


\item \textbf{spec\_type}: (\texttt{string}) type of calculation that will be performed. Supported types:
\begin{itemize}
\item OH stretch, keyword: \texttt{wsOH}
\item OD stretch, keyword: \texttt{wsOD}
\item hydroxyl stretch in water isotope mixtures, keyword: \texttt{wsiso}
\item OH-stretch fundamental-HOH bend overtone, keyword: \texttt{wswbH2O}
\item OD-stretch fundamental-DOD bend overtone, keyword: \texttt{wswbD2O}
\item hydroxyl stretch fundamental-bend overtone in water isotope mixtures, keyword: \texttt{wswbiso}.
\end{itemize}

\item \textbf{D2O}: (\texttt{int}) the number of D$_2$O molecules mixed with H$_2$O. This is only required for \texttt{--spec\_type=wsiso}
and  \texttt{--spec\_type=wswbiso}.

\item \textbf{Fc}: (\texttt{float}) OH-stretch fundamental-HOH bend overtone Fermi coupling. This is  required only for \texttt{--spec\_type=wswbH2O},
\texttt{--spec\_type=wswbD2O}, and \texttt{--spec\_type=wswbiso}.

\item \textbf{DOD\_overtone}: (\texttt{bool}) use this option to turn on/off DOD bend overtone in water hydroxyl stretch fundamental/bend overtone
calculations of isotope mixtures. This option only works with \texttt{--spec\_type=wswbiso}.

\item \textbf{trdipSFG}: (\texttt{float}) distance in \AA{} between O atom and the
OH-stretch transition dipole along the O-H(D) bond. This only works with  
\texttt{--SFG=1}.

\subsection{Spectroscopic maps}

\section{Examples}
\begin{itemize}

\item Generate input files for FTIR calculation of pure H$_2$O:


\end{itemize}

 

\end{itemize}

\begin{appendix}
\section{\label{app:atomsf}atoms\_file}
Below is an example of a file that can be used with the \textbf{atoms\_file} command line. This is a pure water TIP4P simulation:
\begin{verbatim}
# this is a comment line; atom  charge  mass
OW      0.0     16.000
HW1     0.52    1.008
HW2     0.52    1.008
MW      -1.04   0.000
\end{verbatim}

\end{appendix}

\end{document}